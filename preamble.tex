
% \usepackage[draft=false]{graphicx}

\usepackage{booktabs}

\usepackage{longtable}
    \setlength{\LTleft}{0mm} % links ausrichten; gleich wie 'hang' der caption (s. unten)
    
\usepackage{ngerman}

% \usepackage{xcolor} % wird später von kableExtra geladen; gibt sonst clash
\PassOptionsToPackage{table, dvipsnames}{xcolor}

% https://stackoverflow.com/questions/40892802/r-markdown-link-is-not-formatted-blue-when-knitted-to-pdf
% urlcolor: blue % im YAML-Header


\ifxetex
    % https://tex.stackexchange.com/questions/85696/what-causes-this-strange-interaction-between-glossaries-and-amsmath
  \makeatletter % undo the wrong changes made by mathspec
    \let\RequirePackage\original@RequirePackage
    \let\usepackage\RequirePackage
  \makeatother
  % normal weiter
  \usepackage{letltxmacro}
  \setlength{\XeTeXLinkMargin}{1pt}
  \LetLtxMacro\SavedIncludeGraphics\includegraphics
  \def\includegraphics#1#{% #1 catches optional stuff (star/opt. arg.)
    \IncludeGraphicsAux{#1}%
  }%
  \newcommand*{\IncludeGraphicsAux}[2]{%
    \XeTeXLinkBox{%
      \SavedIncludeGraphics#1{#2}%
    }%
  }%
\fi


% weitere Pakete, eigene Erweiterungen und Anpassungen

\usepackage[sfdefault]{roboto}
\usepackage{roboto-mono}




\usepackage{setspace} % Durchschuss

\usepackage{caption}
	\captionsetup{labelfont = bf,
			font = small,
			indention = 4mm,
			labelsep = quad,
			singlelinecheck = false,
			aboveskip = 0.75ex,
			belowskip = 0.75ex}
			
% Kopfzeile
\usepackage{fancyhdr}
    \lhead{Demo R-Bookdown}
    \rhead{\nouppercase{\textit \leftmark}} % no upper case; class 'book'
    % \rhead{\textit \nouppercase{\rightmark} \nouppercase{\leftmark} } % no upper case; class 'book'
    \cfoot{\thepage}
    \renewcommand{\headrule}{\color{gray}\hrule}


% Place floats 'H' (HERE)
\usepackage{float}

\makeatletter
	\setlength{\@fptop}{10pt}
\makeatother
\renewcommand{\floatpagefraction}{0.95}


% Change Numbering accordingly to HTML (per Section)
\usepackage{chngcntr}
  \counterwithin{figure}{section}
  \counterwithin{table}{section}
  
  
% Set background color for code chunks
% https://github.com/rstudio/bookdown/blob/master/inst/examples/latex/preamble.tex
% \definecolor{shadecolor}{RGB}{100, 100, 100}


% Language; muss hier separat angegeben werden...
\renewcommand{\contentsname}{Inhalt}
\renewcommand{\figurename}  {Abbildung}
\renewcommand{\tablename}   {Tabelle}


\usepackage{tikz}

\usepackage[hang, flushmargin, bottom]{footmisc}  % Fussnoten; nach 'setspace' laden


% Bitte keinen Titel plotten
\AtBeginDocument{\let\maketitle\relax}
\setcounter{tocdepth}{2} 


\setstretch{1.05}


% Die ganze Titelseite
\newcommand{\titelseite}{	

	\thispagestyle{empty}
	
    % Kopfzeile Titel
    \begin{tikzpicture}[remember picture, overlay]
    	\node  [xshift = 12 mm, yshift = -17 mm] 
    	at (current page.north west) 
    	[inner sep = 0pt, below left, anchor = north west] { 
    	    \resizebox{128mm}{!} { % \resizebox{width}{height}{object}
    	        \includegraphics{images/Briefkopf.png}
    	        }
    	};  
    \end{tikzpicture} 
    
    \par
    $\phantom{NULL}$
    \vfill
    
    \huge \textbf{Demo R-Bookdown} 
    
    \vspace{1em}
    
    \Large Am Beispiel Bevölkerungsentwicklung  1991\,--\,2045
        
    \vspace{1.5em}
    
    \vfill
    
    \normalsize
    
    Verfasst von Tinu Schneider, Thun 
    
    Erstellt am \today 
    
    
    \thispagestyle{empty}
    
    \newpage
    
    \setstretch{1.03}
    
    
    % Inhalt
    \tableofcontents
    
    \vfill
    
    \thispagestyle{empty}
    % \newpage
}


% Anschliessend das Dokument








